\documentclass[12pt, a4paper]{article}

\usepackage[utf8x]{inputenc}
\usepackage[greek, english]{babel}
\usepackage{caption}
\usepackage[section]{placeins}
\usepackage{balance}
\usepackage{dblfloatfix}
\usepackage{hyperref}

\newcommand{\en}{\selectlanguage{english}}
\newcommand{\gr}{\selectlanguage{greek}}

\title{ΜΗΝΙΑΙΑ ΑΝΑΦΟΡΑ ΠΡΟΓΡΑΜΜΑΤΟΣ \en HOMORE}
\author{Γεώργιος Βαρδάκας}

\begin{document}

\gr
\maketitle

\gr
\noindent
Ξεκινήσαμε με την μελέτη και την κατανόηση του προβλήματος. Το ζητούμενο του έργου είναι για τα δεδομένα που συλλέγονται από τους 7 αισθητήρες του ρολογιού να κατασκευαστούν μέθοδοι αυτόματης κατάτμησης και κατηγοριοποίησής τους. Αυτό επιτυγχάνεται μοντελοποιώντας τα σήματα ως μία πολυδιάστατη χρονοσειρά. Η κατηγοριοποίηση γίνεται με βάση τις δραστηριότητες του χρήστη.
Στο πλαίσιο του \en data exploration \gr έγινε έλεγχος των δεδομένων για τυχών ελλείψεις τιμών (\en missing values\gr) και στην συνέχεια ελέγχθηκε το πλήθος των δειγμάτων ανά κατηγορία, έτσι ώστε να εντοπιστούν τυχόν \en class imbalances\gr.
Στο πλαίσιο της προεπεξεργασίας των δεδομένων (\en data preprocessing\gr) έγινε διαχωρισμός της πολυδιάστατης χρονοσειράς ανά δραστηριότητα και κατασκευάστηκε μηχανισμός διαχωρισμού της χρονοσειράς με επικαλυπτόμενα παράθυρα (\en segments\gr), αναθέτοντας σε κάθε \en segment \gr το αντίστοιχο \en activity id\gr. 


\end{document}